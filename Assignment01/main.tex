\documentclass[fleqn,10pt]{wlscirep}
\usepackage[utf8]{inputenc}
\usepackage[T1]{fontenc}
\usepackage{graphicx}
\title{How to Control Source Code Using Github}

\author{\normalsize 1st Assignment of Mathematical Foundations for Computer Vision and Machine Learning}
\affil{{Dahye Kim 20153712}\\{Chung-Ang University}\\{$20^{th}$ of September of 2018}}


%\keywords{Keyword1, Keyword2, Keyword3}

\begin{abstract}
Github is the most popular and widely used tool for version control. With Github, developers can effectively manage their projects and collaborate in teams. To utilize Github, developers should understand the basic concepts associated with it. This report introduces what is Github and provides basis knowledge needed to use it. Particularly, this document focuses on terminologies used in Github and basic operation commands.
\end{abstract}
\begin{document}

\flushbottom
\maketitle
% * <john.hammersley@gmail.com> 2015-02-09T12:07:31.197Z:
%
%  Click the title above to edit the author information and abstract
%
\thispagestyle{empty}
\section*{1. Introduction of Version Control System}

Version control system (VCS) is a system that records changes to a file or set of files over time so that you can recall specific versions later.\cite{GitIntro} When developers build a program, they bump into many cases where they have to turn back to a certain state of the project. Version control provides a convenient and efficient method to do so with saving a lot of time. For example, when you lose some files or snippets of code, you can simply look at the history of changes recorded in VCS and try to modify your codes referencing the record or set them back to previous states.

It is possible to implement VCS in local computer or with a server both. Github, which we are dealing with in this report, provides Distributed Version Control System (DVCS). In DVCS a server contains a remote repository and clients can freely access the remote repository and interactively modify codes in remote and local repository. Clients can push any changes of local version to the remote repository and pull back the remote repository version to the local version. This system helps clients easily collaborate because all authorized clients can access the remote repository at anytime from anywhere and update the version or revert to a certain version.

\section*{2. Basic Concepts to Know about Github}
\begin{itemize}
\item \textbf{Remote Repository:} A repository created on the Github server, where all files of a project are stored and managed.
\item \textbf{Local Repository:} A repository created on a user’s computer, where the user stores and manage files of a project.
\item \textbf{Clone:} Copying all files on a remote repository to the local repository
\item \textbf{Fork:} Copying all files on a central remote repository of an organization or a team to one’s individual remote repository 
\item \textbf{Origin:} One’s individual remote repository, which can be used as a central remote repository of a small size team
\item \textbf{Upstream:} A central remote repository of an organization or a team
\item \textbf{Branch:} A lightweight movable pointer to certain commits\cite{GitBranch}
\item \textbf{Master:} A default branch which is a main branch of a remote repository
\item \textbf{Commit:} A change of a file recorded by a user
\item \textbf{Push:} Integrating a commit or a group of commits of a local repository into a remote repository
\item \textbf{Pull:} Integrating a commit or a group of commits of a remote repository into a working project on a local repository
\item \textbf{Fetch:} Only downloading new data from a remote repository but not integrating them into the working project on a local repository
\item \textbf{Pull Request:} Asking to pull one’s pushed commits to an upstream repository
\item \textbf{Merge:} Integrating created pull requests into an upstream repository
\end{itemize}


\section*{3. Basic Commands of Github}

\subsection*{1) How to creat a repository on Github and connect the Github repository with a local repository}
\begin{verbatim}

<Local directory>$ git init \\ initializing an empty git repository in the directory
<Local directory>$ git remote add origin <your origin repository url>
<Local directory>$ git push -u origin master

\end{verbatim}

\subsection*{2. How to commit}

\begin{verbatim}

<Local directory>$ git add .  \\ Add all changed files into the staging area
<Local directory>$ git add [some-file] \\ Add some-file into the staging area
<Local directory>$ git commit -m “Write commit message” \\ Create a change record

\end{verbatim}
Shortcut of commands above:
\begin{verbatim}
<Local directory>$ git commit -a -m “Write commit message”

\end{verbatim}


\subsection*{3) How to push}
\begin{verbatim}

<Local directory>$ git push origin [branch name]

\end{verbatim}


\subsection*{4) How to pull}
\begin{verbatim}

<Local directory>$ git pull [origin or upstream] [branch name]

\end{verbatim}


\subsection*{5) How to connect a local repository with a upstream repository}
\begin{verbatim}

<Local directory>$ git remote add upstream [upstream url]

\end{verbatim}


\subsection*{6) How to create branch}
\begin{verbatim}

<Local directory>$ git branch [branch name] \\ Create a new branch
<Local directory>$ git checkout [branch name] \\ Move to specified branch

\end{verbatim}
Shortcut of commands above:
\begin{verbatim}
<Local directory>$ git checkout -b [branch name]

\end{verbatim}


\subsection*{7) How to check working logs}
\begin{verbatim}

<Local directory>$ git reflog

\end{verbatim}

\bibliography{sample}

\section*{Additional Attachment}

Project Github URL: \url{https://github.com/joyfuldahye/MFCVML_assignment01}

\begin{figure}[ht]
\centering
\includegraphics[width=\linewidth]{ScreenShot.png}
\caption{A screen shot of the project repository}
\label{fig:screen}
\end{figure}
\end{document}